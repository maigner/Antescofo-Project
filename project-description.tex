\documentclass[onecolumn,nocopyrightspace,preprint]{sigplanconf}

\usepackage{booktabs}
\usepackage{listings}
\usepackage{hyperref}
\usepackage{xspace}
\usepackage{caption}
\usepackage[tocentry]{vhistory}

% \lstset{
%   commentstyle=\small\ttfamily, %
%   fontadjust=true, %
%   firstnum  ber=1, %
%   escapeinside={(*}{*)}, %
% }

\lstset{
  float=*,
  numbers=none,
  numberstyle=\footnotesize,
  numbersep=4pt,
  basicstyle=\small\ttfamily,
  keywordstyle=\small\ttfamily\bf,
  tabsize=2,
  breaklines=true,
  frame=lines,
  aboveskip=\bigskipamount,
  belowskip=\bigskipamount,
  %belowcaptionskip=\medskipamount,
  language=bash,
  deletekeywords={env, for}
}

\nocaptionrule



\title{Antescofo: Project Title}
\authorinfo{
  Martin~Aigner
}{Computational Systems Group\\University of Salzburg}{firstname.lastname@cs.uni-salzburg.at}

\begin{document}
\maketitle
%\tableofcontents

\section{Foreword} 
Writing this report is part of the requirements of [CLASSNAME] intended to
make students think about combining different topics of education through
a single common property: the use of computers.


Say here, that this is important because...

The goal is to propose a hypothetical project. (to learn this and that)

The cool thing here: this project has actually happened. The drawback: it is no longer a hypothetical project
and therefore, kind of, contradicts the goal of thinking about: ``How would I plan and lead that project?''
Nevertheless, we did plan the project in advance as much as possible and clearly state any changes we have applied 
to \textit{the plan} during the project. Still, we have improvised a lot, e.g., making up on-demand mini lectures on the fly.
The interested reader might wonder if improvisation further contradicts the idea of planning and describing a hypothetical project. 
We don't think so! A teachers ability to adapt to individual student's knowledge, needs, and interests is, in our opinion,
a key quality to have in the education business.

Our goal with this report is simple. We want to enable others to repeat the project under similar circumstances.
The students achieved a great result which can be viewed at \url{https://youtu.be/a_AVsBpvBVo}

\section{Abstract} 

(Context) The Computational Systems Group Salzburg is involved in a research
project on Antescofo, a real-time multimedia system, developed by IRCAM,
Paris. Antescofo is a complex piece of software used to accompany musicians
and orchestras on the stage. It is used at various concert halls throughout
the world, including the Festspielhaus in Salzburg. We have recently submitted
a research proposal with IRCAM on advancing the real-time aspects of Antescofo
for embedded devices.

(Internship) The task of the students within this internship is to setup, use,
(and so performance analysis of Antescofo. Some of the challenges of Antescofo
(are scalability, as well as proper modelling of time, topics that our
(research group has expertise on. The students are expected to get Antescofo
(running in a lab environment, demonstrate simple use with an actual
(instrument, and isolate performance issues that motivate our research). This
(internship project will be a valuable kick-off for our research on enhancing
(the real-time aspects of Antescofo.

(technical bla-bla) Assets for the students: experience working on a highly
(sophisticated software system; get acquainted with technical issues of
(setting up a system; experience with performance analysis and with research
(on real-time aspects of computing; fun with music and complex software.

\section{Tools}
Pure Data, Antescofo, Logic, iMovie

\subsection{Pure Data}

\subsection{Antescofo}

\subsection{Logic}

\subsection{iMovie}


\section{Time Table}

The time frame for the project is approximately 2 weeks, 6 hours per day, or 60 hours.
Note that the time table is heavily affected by the students' prior knowledge in
programming. Table REFERENCE gives a brief overview of the suggested time required for each individual project task.






4 weeks, preparation classes, prerequisites



\end{document}


